% !TEX root = ../main.tex
\section{Introductory Remarks}

\begin{itemize}
\item Optimistic rollups are being used
\item Problem: multi-day (\eg seven) window for withdrawal
\item Why is this a problem? Others are solving this problem (alternative bridges with a TTP). Some examples include: Speculators want to move fast, voting in a DAO (red tape), sell them on L1 or another L2, DApp access on L1 (trading if efficient on L2 but you need to do something on L1). 
\item Disadvantage relative to zk-rollups
\item Solution: scope to liquid tokens (ETH and ERC20), open problem: NFTs or other non-substitutable tokens. 
\item Solution: (1) tradeable exits; (2) market to trade; (3) guaranteed exit (buyer runs ArbOS validator); (4) prediction market solution to guarantee exits to non-validating entities (importantly includes smart contracts)
\item Testing: we implemented (1) and (3); for (2) and (4), use your favourite DeFi project.
\end{itemize}

 
% = = = = = = = = = = = = = = = = = = = = = = = = = = = = = = = = = = = = = = = = = =

\section{Background} 

\begin{itemize}
\item L1 and its scalability issues
\item Rollups as a solution
\item ZK-Rollups vs Optimistic (positive not normative)
\item Bridge works: inbox (few sentences: every L2 tx is on L1, in calldata, sequenced FIFO, sequencer) -> deterministic output, outbox (assertion), eventual finality
\item Withdrawal problem: origins
\end{itemize}

% = = = = = = = = = = = = = = = = = = = = = = = = = = = = = = = = = = = = = = = = = =

\section{Proposed Solution} 

\subsection{Design Landscape}

\begin{itemize}
\item Landscape: atomic swaps
\item Landscape: Change the bridge instead of using a third party contract (reason: too late once you withdrawal)
\end{itemize}

\subsection{Fast Bridge} 

\begin{itemize}
\item Allow trading of exits (atomic unit) -> track most recent owner (constant time) in outbox. Authorization (only current owner can transfer) -> on execute, check for current owner. 
\item Self-insurance (staking a fidelity bond) (you need liquidity) 
\item Prediction market: someone else will insure it
\end{itemize}

\subsection{Implementation} 

\begin{itemize}
\item Arbitrum's Nitro. Bridge: inbox, sequencer, outbox.
\item Modify outbox to allow tradeable exits
\item L1 gas costs: new function (transfer)
\item L1 gas costs: execute: before and after (1 trade, 2 trades, 3 trades) (functions that run that got modified) (difference of two mappings)
\item Unit tests: say something
\item What happens when an assertion fails? (pro: ticket fails with assertion, better for prediction markets (betting on assertion which is a batch of exits); ticket passes even if assertion fails, sounds better (caveat: probably won't happen)
\item Challenges: SDK, where to change, unit test failed assertions (good assertion, bad assertion where withdrawal is ok, bad assertion where the withdrawal is problematic)
\item Expose assertion failures/successes to external contract (submit ID for assertion, get back status: pending, finalized, or discarded). Write down the SDK call. (what happens to a failed assertions???)
\end{itemize}


% = = = = = = = = = = = = = = = = = = = = = = = = = = = = = = = = = = = = = = = = = =

\section{Discussion}

\subsection{Where to Solve Problem? }
\begin{itemize}
\item Exit to a third party collateral contract, implements trading
\end{itemize}

\subsection{Withdrawal Format }
\begin{itemize}
\item Divisible exits
\item ERC20 / ERC721? -> allowances or not? 
\end{itemize}

\subsection{Non-Substitutable Withdrawals}

\begin{itemize}
\item Illiquid tokens 
\item NFTs (substitute for ETH -> support)
\item Messages (oracle updates, read calls, any L2-to-L1 message...) (still offer insurance)
\end{itemize}

\subsection{Assertion Failures}

\begin{itemize}
\item does exit go away with the assertion or not? Pros/cons
\end{itemize}

\subsection{Markets}

\begin{itemize}
\item Can't use an AMM 
\item Can't run out
\item Simple auction
\item Divisible exits
\item Exit pools
\end{itemize}



